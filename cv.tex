\documentclass{article}

%A Few Useful Packages
\usepackage{marvosym}
\usepackage{fontspec} 					%for loading fonts
\usepackage{xunicode,xltxtra,url,parskip} 	%other packages for formatting
\RequirePackage{color,graphicx}
\usepackage[usenames,dvipsnames]{xcolor}
\usepackage{fullpage}
\usepackage{titlesec}					%custom \section

\usepackage[left=0.75in, right=0.75in, top=0.75in, bottom=0.5in]{geometry}
%Setup hyperref package, and colours for links
\usepackage{hyperref}
\definecolor{linkcolour}{rgb}{0,0.2,0.6}
\hypersetup{colorlinks,breaklinks,urlcolor=linkcolour, linkcolor=linkcolour}

%FONTS
\defaultfontfeatures{Mapping=tex-text}
\setmainfont[
SmallCapsFont = Fontin-SmallCaps.otf,
BoldFont = Fontin-Bold.otf,
ItalicFont = Fontin-Italic.otf
]
{Fontin.otf}
\titleformat{\section}{\Large\scshape\raggedright}{}{0em}{}[\titlerule]
\titlespacing{\section}{0pt}{3pt}{3pt}

%-------------WATERMARK TEST [**not part of a CV**]---------------
\usepackage[absolute]{textpos}

\setlength{\TPHorizModule}{30mm}
\setlength{\TPVertModule}{\TPHorizModule}
\textblockorigin{2mm}{0.65\paperheight}
\setlength{\parindent}{0pt}

\input{credentials}

%--------------------BEGIN DOCUMENT----------------------
\begin{document}

\pagestyle{empty} % non-numbered pages

\font\fb=''[cmr10]'' %for use with \LaTeX command

%--------------------TITLE-------------
\par{\centering
		{\Huge Samuel \textsc{de Jong}
	}\bigskip\par}

%--------------------SECTIONS-----------------------------------

\begin{tabular}{l}
    \address \\
    \phone \\
    \email
\end{tabular}

\section{summary}
Through six years of graduate school education and research, I have developed skills in:
\begin{itemize}
	\item
	Creative problem solving and experimental design
	\item
	Developing detector systems and software to acquire and analyze data
	\item
        Quantitative analysis of large data sets
	\item
	Producing data visualizations using matplotlib and ROOT
	\item
	Programming and scripting with C++, python, and bash
	\item
	Collaborating with colleagues locally and internationally
	\item
	Presenting results at conferences and collaboration meetings
	\item
	Writing technical and academic documents
	
\end{itemize}

\section{Experience}
\begin{tabular}{p{14cm}l}
 \textit{PhD Researcher}, University of Victoria & \textsc{Jan 2013 - May 2016} \\
	 \	 \footnotesize{Designed, assembled, and commissioned thermal neutron detector system using tubes of helium-3 for the Belle II experiment. Developed and tested data acquisition system for helium-3 tubes. Oversaw deployment and operation of this detector system in Japan, and analyzed data recorded by them. Calibrated helium-3 tubes at UVic. Simulated helium-3 tubes and neutron source}&\\
&\\

  \textit{MSc Researcher}, University of Victoria & \textsc{Sept 2010 - Aug 2012}\\
	 \footnotesize{Investigated a new technique in particle identification in gaseous detectors. Developed and tested algorithms to implement this new technique.}&\\
&\\

  \textit{Laboratory Instructor}, University of Victoria & \textsc{Sept 2010 - Apr 2016}\\
	 \footnotesize{Instructed undergraduate students on proper laboratory technique and equipment use. Taught experiments in introductory physics, electricity and magnetism, and laboratory electronics. Evaluated student progress and graded lab reports.}&\\
&\\



\end{tabular}


\section{Education}
\begin{tabular}{p{14cm}l}	
  Doctorate in \textsc{High Energy Physics}, \textbf{University of Victoria} & \textsc{May} 2017 \\
 Thesis: ``Thermal Neutrons in Phase I of Belle II Commissioning'' &\\
 \small Advisor: Prof. Michael Roney&\smallskip\\


 Masters of Science in \textsc{High Energy Physics}, \textbf{University of Victoria} & \textsc{August} 2012\\
 Thesis: ``Cluster Counting Studies in a SuperB Drift Chamber Prototype'' &\\
 \small Advisor: Prof. Michael Roney&\smallskip\\


 Bachelor of Science in \textsc{Applied Physics} with \textsc{Mathematics} minor, \normalsize\textbf{Carleton University} & \textsc{December} 2009\\

\end{tabular}

\section{Skills}
\textbf{Computing} \\
\begin{tabular}{r|l}
Advanced Knowledge&\textsc{C++}, python, \textsc{ROOT}, \textsc{Linux}, Object Oriented Programming \\
Intermediate Knowledge& GEANT4, \textsc{java}, Excel, EPICS, \textsc{Windows}, ubuntu, SL6, \textsc{bash},  {\fb \LaTeX}\setmainfont[SmallCapsFont=Fontin-SmallCaps.otf]{Fontin.otf}, svn, git, Virtual Machines,\\& \textsc{jupyter}, \textsc{xml}, Confluence \\
Basic Knowledge& \textsc{html}, \textsc{MATLAB}, matplotlib, SQL\\

\end{tabular}

\textbf{Hardware}\\
\begin{tabular}{r|l}
 Expertise in & Detector development, DAQ software development \\	
 Equipment & \textsc{Caen}, \textsc{VME}, \textsc{NIM}, Digitizers, Computer-Electronics interfaces, Power supplies \\
\end{tabular}

\textbf{Other}\\
\begin{tabular}{l}
 Particle Physics, Data analysis, Statistics, Public speaking, Problem solving, Collaborative research, Teaching\\	
\end{tabular}
	

\section{Conferences and Presentations}

During my PhD studies, I travelled to the KEK lab in Tsukuba, Japan for meetings of the Belle II collaboration on five separate occasions, and presented remotely many other times. At these meetings, I presented the status of the project that I had been working on to colleagues and answered questions about my project.



\begin{tabular}{p{14cm}l}
 \textit{Oral Presentation}, Canadian Association of Physics Annual Congress (Ottawa, ON) & 2016\\ \footnotesize{Presented results of experiments conducted at the KEK physics lab in Tsukuba, Japan.} &\smallskip\\

 \textit{Oral Presentation}, Winter Nuclear and Particle Physics Conference (Montreal, QC) & 2012\\ \footnotesize{Presented results of experiments conducted at the TRIUMF physics lab in Vancouver, BC.} & \smallskip\\

 \textit{Oral Presentation}, Canadian Association of Physics Annual Congress (Calgary AB)& 2012\\ \footnotesize{Presented results of experiments conducted at the TRIUMF physics lab in Vancouver, BC.} & \smallskip\\

\textit{Poster Presentation}, University of Victoria (Victoria BC) & 2011 \\ \footnotesize{Presented an introduction to the SuperB experiment.} &\smallskip \\

 \textit{Oral Presentation}, Canadian Undergraduate Physics Conference (Edmonton AB)& 2009\\ \footnotesize{Presented a summary of studies I had done for my BSc Honours project.} & \smallskip\\
\end{tabular}

\section{Field work}

\begin{tabular}{p{14cm}l}
 \textbf{TRIUMF}, Vancouver, Canada: Prototype drift chamber at M11 & 2011\\ \footnotesize{Performed shift work for recording data, adjusted beam settings to produce different particle momenta, changed the position of the prototype chamber, problem solved issues with the experiment} &\smallskip\\

 \textbf{KEK}, Tsukuba, Japan: BEAST II Phase I & 2016\\ \footnotesize{Installed helium-3 tubes in the SuperKEKB commissioning detector, BEAST II. Performed data acquisition shifts, attended accelerator meetings, problem solved issues with the experiment} &\smallskip\\


\end{tabular}



\section{Publications}
\begin{tabular}{p{14cm}l}
  \textit{First Measurements of Beam Backgrounds at SuperKEKB} & 2017\\
	 Vahsen, S., et al. &\smallskip\\



 \textit{Improved particle identification using cluster counting in a full-length drift chamber prototype} & 2014\\
Caron, J.-F., Hearty, C., Lu, P., So, R., Cheaib, R., Martin, J.-P., Faszer, W., \textbf{de Jong, S.}, Beaulieu, A., Roney, M., de Sangro, R., Felici, G.,  Finocchiaro, G., and Piccolo, M. &\smallskip\\


 \textit{SuperB Technical Design Report} & 2013\\
	{Baszczyk, M., et al.} &\smallskip\\

\textit{Prospects for Observing the Standard Model Higgs Boson Decaying into $b\bar{b}$ Final States} & 2010 \\ \textit{Produced in Weak Boson Fusion with an Associated Photon at the LHC} &\\
 {Asner, D.M., Cunningham, M., \textbf{de Jong, S.}, Randrianarivony, K., and Santamarina, C. }&\\


\end{tabular}



\section{Interests and Activities}
\textbf{Professional:}
Problem solving, Data analysis, Programming, Data acquisition systems, Hardware-software interface, Physics, Experimentation, Detector development, Simulation of detector systems, Research and development\\
\textbf{Personal:}
Photography, Science Fiction, Travel

\vspace{\fill}
\centering Samuel \textsc{de Jong} 2/2




\end{document}
